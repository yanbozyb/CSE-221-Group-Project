
\begin{table*}[t]
	\centering
	\begin{tabular}{c|c|c|c|c|c|c}
		\hline
		\textbf{Category} & \textbf{Operation} & \textbf{Module} & \textbf{Hardware} & \textbf{Software}  & \textbf{Predicted} & \textbf{Measured} \\ \hline \hline

		\multirow{4}{*}{CPU, Scheduling, OS} & Measurement & xx & xx & xx & xx & xx \\ \cline{2-7}

          & Procedure call & xx & xx & xx & xx & xx \\ \cline{2-7}
          
          & Task creation & xx & xx & xx & xx & xx \\ \cline{2-7}
        
          & Context switch & xx & xx & xx & xx & xx \\ \hline

        \multirow{6}{*}{Memory} & \multirow{4}{*}{RAM access} & L1 & 1ns & Non & 1ns & 1.5ns \\ \cline{3-7}
                                &  & L2 & 4ns & Non & 4ns & 5ns \\ \cline{3-7}
                                &  & L3 & 30ns & Non & 30ns & 25ns \\ \cline{3-7}
                                &  & DRAM & 90ns & 5-10ns & 95-100ns & 110ns \\ \cline{2-7}

          & \multirow{2}{*}{RAM bandwidth} & Read & 12.8 GB/s & Small & 12.0 GB/s &  11.36 GB/s \\ \cline{3-7}

          & & Write & 12.8 GB/s & Small & 12.0 GB/s & 7.64 GB/s \\ \cline{2-7}

          & Page fault & - & 70-90us & Small & 70-90us & 27us \\ \hline

          \multirow{3}{*}{Network} & Round trip & xx & xx & xx & xx & xx \\ \cline{2-7}

         & Peak bandwidth & xx & xx & xx & xx & xx \\ \cline{2-7}

         & Connection overhead & xx & xx & xx & xx & xx \\ \hline

         \multirow{4}{*}{File System} & File cache & - & 90ns & 1us & 1us & 1us \\ \cline{2-7}
        
         & File read & - & 70-90us & 10us & 80-100us & 400us \\ \cline{2-7}

         & Remote file read & - & 300ms & 10us & 300ms & 300ms \\ \cline{2-7}

         & Contention & 16 process & 300-500us & 10us & 500us & 2200us \\ \hline
	\end{tabular}
	\caption{\textbf{Experimental Results Summary.} Hardware (time) is from multiple specification of products~\cite{memorytime,memorytime,memorybandwidth}. Software is estimated time based on our knowledge. Predicted and measured are end-to-end time from our prediction and real experiments respectively.}
	\label{table:summary}
	%\vspace{-7mm}
\end{table*}
\section{Summary}
\label{sec:summary}
In this section, we summarize our experiments and findings, and compare them with hardware as well as our estimated results. In Table~\ref{table:summary}, we summarize all hardware performance, estimated software overhead, our predicted time, and measured time of our experiments.

\paragraph{Xiaochuan: CPU, Scheduling, OS}

\paragraph{Memory operations.} For memory operations, we can see that both RAM access and bandwidth demonstrate similar performance results as we predicted. DRAM access time shows a bit of higher than we thought and RAM write bandwidth shows a small part of lower results than we thought. We conclude that these light deviation is because our machine is a virtualized environment, the real hardware is different with our predicted hardware; and virtualization will also introduce a little overhead which causes the unexpected results. One difference is page fault time which is pretty lower than our predicted time. We think this may be caused by the memory hierarchy. In our test machine, there is a L3 shared cache across all CPU cores. Therefore, when page fault is triggered, the OS may use L3 to handle that (e.g., prefetch data from DRAM to L3 cache in advance). 

\paragraph{Xiaochuan: Network}

\paragraph{Filesystem operations.} For filesystem operations, we can see that our estimated time of file cache and remote file read time are is similar to the results measured from the experiments. However, the file read time (local access) and contention results are much higher than what we thought. For file read, we think that the difference is because of the virtualized environment, we can not exactly estimate the underlying device time and software overhead. In virtualized environment, the software overhead depends on multiple components including hypervisor, virtualization technologies (e.g., full-virtualization, para-virtualization), and the underlying storage software cloud providers used. For contention under multiple processes, the measured results demonstrate that OS would lead to more overhead caused by multiprocess synchronization, which is beyond our imagination.